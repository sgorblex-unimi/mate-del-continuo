%% Copyright (C) 2019-2021 Alessandro Clerici Lorenzini
%
% This work may be distributed and/or modified under the
% conditions of the LaTeX Project Public License, either version 1.3
% of this license or (at your option) any later version.
% The latest version of this license is in
%   http://www.latex-project.org/lppl.txt
% and version 1.3 or later is part of all distributions of LaTeX
% version 2005/12/01 or later.
%
% This work has the LPPL maintenance status `maintained'.
%
% The Current Maintainer of this work is Alessandro Clerici Lorenzini
%
% This work consists of the files listed in work.txt


% TODO: cosa succede se l'argomento dei simboli è 0?
Ai fini di semplificare e risolvere problemi nell'ambito dei limiti di successioni, è utile introdurre una nuova terminologia, sviluppata dal matematico tedesco Edmund Landau.

\section{\texorpdfstring{$o(\cdot)$ (o piccolo)}{o piccolo}}
Il primo simbolo è quello di $o(a_n)$ ($o$ piccolo di $a_n$), che viene definito come segue:
\begin{defin}
	\[
		o(b_n)=\left\{a_n \mid \frac{a_n}{b_n}\to0\right\}
	\]
	Inoltre, per notazione:
	\[
		a_n=o(b_n)\iff a_n\in o(b_n)
	\]
\end{defin}
Questo simbolo definisce una relazione, ma non un'uguaglianza. Questa relazione non gode infatti di due proprietà tipiche delle relazioni di equivalenza, la proprietà riflessiva e quella simmetrica (come si può dimostrare con semplici controesempi) mentre gode della proprietà transitiva:
\begin{prop}[Transitiva]
	\label{optrans}
	\[
		\forall a_n,b_n,c_n:
		\begin{cases}
			a_n=o(b_n) \\
			b_n=o(c_n)
		\end{cases}\Rightarrow a_n=o(c_n)
	\]
\end{prop}
\begin{proof}
	Per definizione, bisogna dimostrare che:
	\[
		\frac{a_n}{c_n}\to0
	\]
	Moltiplicando e dividendo per $b_n$:
	\[
		\frac{a_n}{b_n}\cdot \frac{b_n}{c_n}
	\]
	Per ipotesi entrambi i fattori tendono a $0$, quindi il limite del prodotto è $0$.
\end{proof}

\subsection{Proprietà e relazioni}
Un "uguaglianza" tra due $o$ non è riflessiva né simmetrica. In particolare:
\[
	o(a_n)=o(b_n)
\]
significa che ogni elemento dell'insieme $o(a_n)$ appartiene all'insieme $o(b_n)$, ovvero che l'insieme $o(a_n)$ è contenuto nell'insieme $o(b_n)$:
\[
	o(a_n)\subseteq o(b_n)
\]

\subsubsection{Somma di un $o$ e una successione}
Si introduce ora quello che si può considerare l'analogo per $o$ del primo principio di equivalenza delle equazioni.
\begin{teor}
	\[
		o(a_n)=b_n+o(c_n)\iff o(a_n)-b_n=o(c_n)
	\]
\end{teor}
\begin{proof}
	Si nota che il secondo membro (della prima proposizione) è la somma di un elemento con un insieme, cioè l'insieme delle somme di $b_n$ con ciascun elemento $\delta_n$ dell'insieme $o(c_n)$:
	\[
		b_n+o(c_n) = \{b_n+\delta_n\mid\delta_n\in o(c_n)\}
	\]
	Quindi la relazione iniziale si traduce nel dire che se un elemento $\gamma_n$ appartiene a $o(a_n)$, esso appartiene anche all'insieme dei $b_n+\delta_n$ (per quanto detto nel paragrafo precedente), ovvero esiste un $\delta_n$ in $o(c_n)$ tale che $\gamma_n=b_n+\delta_n$, dove l'ultimo $=$ ha il significato ordinario (non essendo applicato a $o$):
	\[
		\gamma_n=o(a_n)\Rightarrow\gamma_n=b_n+o(c_n)\Rightarrow\exists\delta_n=o(c_n)\mid\gamma_n=b_n+\delta_n
	\]
	E ciò vale per ogni $\gamma_n$ appartenente a $o(a_n)$. Essendo l'ultima un'equazione "classica" e quindi simmetrica, si può applicare:
	\[
		\delta_n=\gamma_n-b_n
	\]
	Quindi se $\delta_n$ appartiene a $o(c_n)$, significa che
	\[
		\gamma_n=o(a_n)\Rightarrow\gamma_n-b_n=o(c_n)
	\]
	In virtù del fatto che la relazione vale per ogni $\gamma_n$ che appartenga a $o(a_n)$, si può utilizzare $o(a_n)$ per indicare ogni elemento che vi appartiene:
	\[
		o(a_n)-b_n=o(c_n)
	\]
	% TODO: spiegare meglio
	Che tradotto usando la definizione significa anche
	\[
		\frac{\gamma_n}{a_n}\to0\Rightarrow\frac{\gamma_n-b_n}{c_n}\to0
	\]
	Analogo vale per l'implicazione inversa.
\end{proof}

\subsubsection{Prodotto di $o$ e una successione}
Esiste anche un analogo del secondo principio di equivalenza delle equazioni:
\[
	o(a_n)=b_n\cdot o(c_n)\iff\frac{o(a_n)}{b_n}=o(c_n)
\]


\subsection{Operazioni}

\subsubsection{Somma di $o(a_n)$ con se stesso}
\begin{teor}
	\[
		o(a_n)+o(a_n)=o(a_n)
	\]
\end{teor}
\begin{proof}
	Questa espressione equivale a dire, per la definizione, che la somma di due successioni $\gamma_{1_n}$ e $\gamma_{2_n}$ che divise per $a_n$ tendono a zero, a sua volta, se divisa per $a_n$, tende a zero. In simboli:
	\[
		\begin{cases}
			\dfrac{\gamma_{1_n}}{a_n}\to0 \\\\
			\dfrac{\gamma_{2_n}}{a_n}\to0 \\
		\end{cases}\Rightarrow\frac{\gamma_{1_n}+\gamma_{2_n}}{a_n}\to0\qquad\forall\gamma_{1_n},\gamma_{2_n}
	\]
	La dimostrazione è quasi immediata:
	\[
		\frac{\gamma_{1_n}+\gamma_{2_n}}{a_n}=\frac{\gamma_{1_n}}{a_n}+\frac{\gamma_{2_n}}{a_n}
	\]
	Poiché entrambi gli addendi tendono a $0$ per ipotesi, la somma tende a $0$, quindi $o(a_n)+o(a_n)=o(a_n)$.
\end{proof}

\subsubsection{Prodotto per una costante}
Come è facile dimostrare, i seguenti insiemi sono uguali:
\[
	o(c\cdot a_n)\qquad c\cdot o(a_n)\qquad o(a_n)
\]
\begin{proof}
	Si utilizza $o(a_n)$ per indicare ogni suo elemento.
	\begin{description}
		\item[$\bullet~ o(c\cdot a_n)=o(a_n)$] A partire dalla definizione:
			\[
				\frac{o(a_n)}{c\cdot a_n}\to 0\qquad\Rightarrow\qquad\frac{o(a_n)}{a_n}\cdot \frac{1}{c}\to0
			\]
			Nella seconda espressione, il secondo fattore è costante, quindi il primo tende a zero.
		\item[$\bullet~ c\cdot o(a_n) = o(a_n)$]
			\[
				c\cdot \frac{o(a_n)}{a_n}\to0
			\]
			Il primo fattore è una costante e il secondo tende a zero per definizione: il limite è zero.
		\item[$\bullet$] Per definizione, da $o(a_n)$:
			\[
				\frac{o(a_n)}{a_n}\to0
			\]
			Si può moltiplicare o dividere l'espressione per una costante, in quanto il limite rimane invariato. È possibile quindi svolgere un'operazione inversa rispetto alle due dimostrazioni precedenti e ottenere le espressioni che definiscono gli altri due insiemi.
	\end{description}
\end{proof}

\subsubsection{Differenza}
La differenza di due $o(a_n)$ non è altro che la somma del primo per il secondo moltiplicato per $-1$. Quindi, in virtù delle dimostrazioni precedenti:
\[
	o(a_n)-o(a_n)=o(a_n)+-1\cdot o(a_n)=o(a_n)+o(a_n)=o(a_n)
\]

\subsubsection{Prodotto per una successione}
I seguenti insiemi sono uguali:
\[
	b_n\cdot o(a_n)\qquad o(a_n\cdot b_n)
\]
pur non essendo uguali a $o(a_n)$.
\begin{proof} ~
	\begin{itemize}
		\item $b_n\cdot o(a_n)=o(a_n\cdot b_n)$:
		      \[
			      \frac{b_n\cdot o(a_n)}{a_n\cdot b_n}\to0\qquad\text{ovvero}\qquad\frac{o(a_n)}{a_n}
		      \]
		      Che tende a $0$ per definizione.
		\item $o(a_n\cdot b_n)=b_n\cdot o(a_n)$:
		      Utilizzando il "secondo principio":
		      \[
			      \frac{o(a_n\cdot b_n)}{b_n}=o(a_n)
		      \]
		      % TODO: ???
		      Ovvero, tradotto in limiti:
		      \[
			      \frac{o(a_n\cdot b_n)}{b_n\cdot a_n}\to0
		      \]
		      che è vero per definizione.
	\end{itemize}
\end{proof}

\subsubsection{$o(1)$ ($o$ piccolo di $1$)}
$o(1)$ è uno strumento che può tornare utile alla semplificazione di calcoli e nelle dimostrazioni. Infatti:
\begin{gather*}
	o(1)=\{a_n\mid a_n\to0\}\\
	o(a_n)\equiv o(a_n\cdot 1)\equiv a_n\cdot o(1)
\end{gather*}
Dove il simbolo $\equiv$ significa "equivale a"

Ecco un esempio di implementazione di questa proprietà:
\begin{examp}
	Stabilire per quali valori di $\alpha$ vale $o(n^\alpha)=o(n \ln n)$. Traducendo l'espressione tramite definizione:
	\[
		\frac{o(n^\alpha)}{n\ln n}\to0
	\]
	Traducendo in espressioni equivalenti:
	\[
		\frac{o(n^\alpha)}{n\ln n}\equiv\frac{n^\alpha}{n\ln n}\cdot o(1)\equiv\frac{n^{\alpha-1}}{\ln n}\cdot o(1)
	\]
	Per $\alpha>1$ il primo fattore tende a $+\infty$, producendo una forma di indecisione, mentre per $a\leq1$ il primo fattore tende a $0^+$, per cui moltiplicato per $o(1)$ tende a $0$.
\end{examp}

\subsubsection{$o(\cdot)$ composti}
Per la proprietà transitiva della relazione definita da $o$ (dimostrata come proprietà \vref{optrans}), vale la relazione:
\[
	o(o(a_n))=o(a_n)
\]
