%% Copyright (C) 2019-2021 Alessandro Clerici Lorenzini
%
% This work may be distributed and/or modified under the
% conditions of the LaTeX Project Public License, either version 1.3
% of this license or (at your option) any later version.
% The latest version of this license is in
%   http://www.latex-project.org/lppl.txt
% and version 1.3 or later is part of all distributions of LaTeX
% version 2005/12/01 or later.
%
% This work has the LPPL maintenance status `maintained'.
%
% The Current Maintainer of this work is Alessandro Clerici Lorenzini
%
% This work consists of the files listed in work.txt


\section{\texorpdfstring{$O$, $\Omega$ e $\Theta$}{O, Omega e Theta} grandi}
I cosiddetti simboli grandi di Landau vengono definiti non più con un limite, ma con un intervallo limitante, nei modi che seguono:
\begin{defin}
	\begin{align*}
		\bullet~ & a_n=O(b_n):\quad\exists:                &       & \abs{\frac{a_n}{b_n}}\leq C\quad & \text{definitivamente} \\
		\bullet~ & a_n=\Omega(b_n):\quad\exists c>0:       &       & \abs{\frac{a_n}{b_n}}\geq c\quad & \text{definitivamente} \\
		\bullet~ & a_n=\Theta(b_n):\quad\exists 0<c\leq C: & c\leq & \abs{\frac{a_n}{b_n}}\leq C\quad & \text{definitivamente}
	\end{align*}
\end{defin}

\subsection{Proprietà}
Come è immediato dimostrare, poiché una successione è asintotica a se stessa, tutte le precedenti relazioni godono della proprietà riflessiva. Diverso è per la proprietà simmetrica:
\begin{gather*}
	a_n=O(b_n)\overset{?}{\Rightarrow} b_n=O(a_n)\\
	\exists C<+\infty: \abs{\frac{a_n}{b_n}}\leq C\quad\text{definitivamente}\\
	\text{ovvero}\\
	\exists C<+\infty: \abs{\frac{b_n}{a_n}}\geq \frac{1}{C}\quad\text{definitivamente}
\end{gather*}
L'ultima proposizione corrisponde con la definizione di $b_n=\Omega(a_n)$ (dove $1/C>0$ poiché $C\in\R$), infatti:
\[
	a_n=O(a_n)\iff b_n=\Omega(a_n)
\]
Lo stesso vale per $\Omega$, mentre per $\Theta$:
\begin{gather*}
	a_n=\Theta(a_n)\iff b_n=\Theta(a_n)\\
	\exists 0<c\leq C<+\infty: c\leq\abs{\frac{a_n}{b_n}}\leq C\quad\text{definitivamente}\\
	\text{ovvero}
	\exists 0<c\leq C<+\infty: \frac{1}{c}\geq\abs{\frac{b_n}{a_n}}\geq\frac{1}{C}\quad\text{definitivamente}
\end{gather*}
Che rispetta ancora la definizione di $a_n=\Theta(b_n)$.

La proprietà transitiva è invece propria di tutti i simboli grandi di Landau:
\begin{prop}
	\[
		\begin{cases}
			a_n=\Omega(b_n) \\
			b_n=\Omega(c_n)
		\end{cases}\Rightarrow a_n=\Omega(c_n)
	\]
\end{prop}
\begin{proof}
	\begin{gather*}
		\exists c>0: \abs{\frac{a_n}{b_n}}\geq c\quad\text{definitivamente}\\
		\abs{\frac{a_n}{b_n}}=\abs{\frac{a_n}{b_n}}\abs{\frac{b_n}{c_n}}
	\end{gather*}
	Il primo fattore è maggiore definitivamente di un valore $c_1$ per ipotesi, e il secondo di un valore $c_2$. Il loro prodotto sarà quindi maggiore del valore $c_1*c_2$ definitivamente, il che significa che $a_n=\Omega(c_n)$.
\end{proof}
\begin{examp}
	\[
		n^2+n\overset{?}{=}O(n^2)
	\]
	Per dimostrare la relazione, bisogna trovare un $C<+\infty$ per cui la seguente disuguaglianza è verificata definitivamente:
	\begin{align*}
		\abs{\frac{n^2+n}{n^2}} & \leq C   \\
		\frac{n^2+n}{n^2}       & \leq C   \\
		1+\frac{1}{n}           & \leq C   \\
		\frac{1}{n}             & \leq C-1
	\end{align*}
	Poiché $1/n$ è sempre minore o uguale a $1$, è sufficiente l'esempio $C=2$ per dimostrare che esiste un $C$ consono alla definizione:
	\[
		\frac{1}{n}\leq 2-1
	\]
	Quanto al problema di determinare tutte le possibili $C$, si riscrive la disequazione nel seguente modo:
	\[
		1\leq(C-1)n
	\]
	La quale si comporta diversamente a seconda del valore di $C$:
	\begin{align*}
		\bullet~ & C>1:\quad & n & \geq\frac{1}{C-1}\quad & \text{definitivamente vera}  \\
		\bullet~ & C=1:\quad & 1 & \leq0\quad             & \text{falsa}                 \\
		\bullet~ & C<1:\quad & n & \leq\frac{1}{C-1}\quad & \text{definitivamente falsa}
	\end{align*}
	Quindi il primo caso è l'unico in cui la condizione è verificata.
\end{examp}

Un teorema molto utile nel caso in cui il rapporto di due successioni tenda a un limite (ma che non dice niente in caso contrario) è il seguente:
\begin{teor}
	Posto:
	\[
		\abs{\frac{a_n}{b_n}}\to L
	\]
	Allora:
	\begin{align*}
		\bullet~ a_n=O(b_n)      & \iff L<+\infty   & \text{con }C\geq L \\
		\bullet~ a_n=\Omega(b_n) & \iff L>0         & \text{con }c\leq L \\
		\bullet~ a_n=\Theta(b_n) & \iff 0<L<+\infty &
	\end{align*}
\end{teor}
\begin{proof}
	Si dimostra il teorema per $\Omega$. Le dimostrazioni per gli altri simboli sono analoghe.
	\begin{itemize}
		\item $L>0 \Rightarrow a_n=\Omega(b_n)$: poiché $\abs{\frac{a_n}{b_n}}$ ammette limite e questo è maggiore di $0$, è sufficiente prendere la parte sinistra di un intorno di $L$ che non includa $0$ e scegliervi un $c$ per dimostrare che $0<c\leq\abs{\frac{a_n}{b_n}}$ definitivamente (per definizione di limite la successione appartiene definitivamente all'intorno - esattamente come nel teorema di permanenza del segno).
		\item $a_n=\Omega(b_n)\Rightarrow L>0$: poiché $\abs{\frac{a_n}{b_n}}\geq c>0$ definitivamente e poiché per ipotesi $\abs{\frac{a_n}{b_n}}$ ammette limite, questo limite dev'essere maggiore di $c$ e di conseguenza anche maggiore di $0$.
	\end{itemize}
	Le due proposizioni si implicano dunque a vicenda.
\end{proof}


\subsubsection{$O$ grande}
Si noti che $O(1)$ non è altro che l'insieme delle successioni limitate (definitivamente, e quindi essendo discrete anche totalmente), in quanto è l'insieme delle successioni comprese definitivamente tra $-C$ e $C$ (ovvero, il cui modulo è minore o uguale a $C$). Per $O$ valgono molte delle proprietà che valgono per $o$. Eccone alcune:
\begin{prop}
	\[
		a_n\cdot O(b_n)=O(a_n\cdot b_n)
	\]
\end{prop}
\begin{proof}
	\begin{gather*}
		\abs{\frac{a_n\cdot O(b_n)}{a_n\cdot b_n}}\leq C\\
		\abs{\frac{O(b_n)}{b_n}}\leq C
	\end{gather*}
	Che è vera per definizione.
\end{proof}
Analogamente valgono:
\begin{itemize}
	\item $O(a_n\cdot b_n)=a_n\cdot O(b_n)$
	\item $\frac{1}{a_n}\cdot O(b_n)=O(b_n)$
	\item $O(a_n)\cdot O(b_n)\equiv O(a_n\cdot b_n)$
	\item $O(a_n)\equiv a_n\cdot O(1)$
\end{itemize}
E, per quanto riguarda la somma:
\begin{prop}
	\[
		O(a_n)+O(a_n)=O(a_n)
	\]
\end{prop}
\begin{proof}
	\begin{gather*}
		\abs{\frac{O(a_n)+O(a_n)}{a_n}}\leq C\quad\text{definitivamente}\\
		\text{per disuguaglianza triangolare:}\\
		\abs{\frac{O(a_n)}{a_n}+\frac{O(a_n)}{a_n}}\leq \abs{\frac{O(a_n)}{a_n}}+\abs{\frac{O(a_n)}{a_n}}
	\end{gather*}
	Se ogni addendo è minore o uguale a un certo $c$, la somma sarà minore o uguale a $2c$.
\end{proof}


\subsubsection{$\Omega$ e $\Theta$}
Omega grande si comporta similmente rispetto alla moltiplicazione ma diversamente per la somma. Infatti, la disuaguaglianza triangolare utilizzata nell'ultima dimostrazione non può avere luogo per $\Omega$, visto l'opposto verso della disequazione. Ecco un semplice controesempio che dimostra la falsità di $\Omega(a_n)+\Omega(a_n)=\Omega(a_n)$:
\begin{gather*}
	\begin{cases}
		n=\Omega(n) \\
		-n+1=\Omega(n)
	\end{cases}\\
	n+(-n+1)=\Omega(n)+\Omega(n)\overset{?}{=}\Omega(n)\\
	1\neq\Omega(n)\quad\text{in quanto tende a $0$}
\end{gather*}

Proprietà del tutto analoghe valgono per $\Theta$. Per fare un analogia con la limitatezza in $O$, che evita un intorno di $+\infty$, si può dire che $\Omega$ evita invece un intorno di $0$.

Riguardo i simboli grandi di Landau vale per le successioni la seguente proprietà:
\begin{prop}
	\label{suc:glelandau}
	Siano $a_n$ e $b_n$ due successioni.
	\begin{itemize}
		% TODO: in modulo? Controesempio: x >= -x^2
		\item $a_n\geq b_n \text{ definitivamente}\quad\Rightarrow\quad a_n=\Omega(b_n)\land b_n=O(a_n)$
		\item $a_n\leq b_n \text{ definitivamente}\quad\Rightarrow\quad a_n=O(b_n)\land b_n=\Omega(a_n)$
		\item $a_n=b_n \text{ definitivamente}\quad\Rightarrow\quad a_n\sim b_n$
	\end{itemize}
\end{prop}
\begin{proof}
	Per $a_n\geq b_n$ vale, definitivamente:
	\[
		\abs{\frac{a_n}{b_n}}\geq 1
	\]
	Poiché $1>0$ la condizione soddisfa la definizione di $\Omega$ (e inversamente di $O$). La seconda proprietà non è altro l'inversa della prima. Per quanto riguarda la terza, vale, definitivamente:
	\[
		\frac{a_n}{b_n}=1
	\]
	Quindi il limite dell'espressione è $1$, da cui l'asintoticità.
\end{proof}


\subsection{Confronto di simboli grandi e piccoli}
Valgono le seguenti proprietà
\begin{prop} \label{prop:landaugp} ~
	\begin{enumerate}
		\item $o(a_n)=O(a_n)$ in quanto se una successione tende a $0$ è limitata. Non vale chiaramente l'inverso, in quanto una successione limitata può, ad esempio, tendere a un numero diverso da $0$.
		\item Per quanto riguarda le potenze di $a_n$:
		      \begin{gather*}
			      O(a_n^\alpha)\overset{?}{=}o(a_n^\beta)\\
			      \frac{O(a_n^\alpha)}{a_n^\beta}=\frac{a_n^\alpha}{a_n^\beta}O(1)=\frac{O(1)}{a_n^{\beta-\alpha}}
		      \end{gather*}
		      La relazione è verificata solo per $\beta>\alpha$.
		\item $O(o(a_n))=o(a_n)$ in quanto
		      \[
			      \frac{o(a_n)\cdot O(1)}{a_n}
		      \]
		      è il prodotto di una successione limitata e zero, e tende quindi a $0$. È sicuramente vero che $O(o(a_n))=O(a_n)$ (per ragioni appena dimostrate), ma in questo caso si va a perdere precisione.
		\item Una somma del tipo $o(a_n)+O(a_n)$ è uguale a $O(a_n)$, con perdita di informazione. Ci si può chiedere se $o(a_n)+O(a_n)=o(a_n)$:
		      \[
			      \frac{o(a_n)+O(a_n)}{a_n}=\frac{o(a_n)}{a_n}+\frac{O(a_n)}{a_n}
		      \]
		      La relazione è vera solo se il secondo addendo tende a $0$. Questa condizione dipende dalla successione $a_n$ e va quindi discussa per caso.
		\item $a_n\sim b_n\Rightarrow a_n=\Theta(b_n)$ in quanto se $\frac{a_n}{b_n}\to1$ la successione rapporto è distante sia da $0$ che da $+\infty$;
		\item $a_n\sim b_n\Rightarrow O(a_n)\equiv O(b_n)$ e $a_n\sim b_n\Rightarrow \Omega(a_n)\equiv \Omega(b_n)$
	\end{enumerate}
\end{prop}
