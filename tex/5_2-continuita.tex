%% Copyright (C) 2019-2021 Alessandro Clerici Lorenzini
%
% This work may be distributed and/or modified under the
% conditions of the LaTeX Project Public License, either version 1.3
% of this license or (at your option) any later version.
% The latest version of this license is in
%   http://www.latex-project.org/lppl.txt
% and version 1.3 or later is part of all distributions of LaTeX
% version 2005/12/01 or later.
%
% This work has the LPPL maintenance status `maintained'.
%
% The Current Maintainer of this work is Alessandro Clerici Lorenzini
%
% This work consists of the files listed in work.txt


\section{Continuità}
La nozione di continuità di una funzione in un punto viene definita come segue:
\begin{defin}
	\label{defin:continua}
	Una funzione $f(x)$ si dice continua in un punto $x_0$ del suo dominio se
	\[
		\lim_{x\to x_0} f(x) = f(x_0)
	\]
	Una funzione continua in tutti i punti del suo dominio si dice semplicemente continua.
\end{defin}
Parafrasando, si può dire che se una funzione si comporta nei pressi di $x_0$ così come in $x_0$ stesso, allora essa è continua in quel punto. Chiaramente, il limite dev'essere finito, infatti una funzione non può assumere i valori $\pm\infty$.


\subsection{Proprietà}
Una proprietà straordinariamente utile nel calcolo dei limiti è quella che riguarda il cambio di variabili:
\begin{prop}
	\label{lim:var}
	\[
		\begin{cases}
			\displaystyle\lim_{x\to x_0} g(x)=y_0 \\
			\displaystyle\lim_{y\to y_0} f(y)=z_0 \\
			g(x)\neq y_0
		\end{cases}\Rightarrow
		% TODO: chiarire per quali x valga la terza condizione
		\lim_{x\to x_0} f(g(x))=z_0
	\]
\end{prop}
\begin{proof}
	Per la definizione di limite:
	\begin{gather*}
		\forall W\in I(y_0) ~\exists V\in I(x_0)\mid g(x)\in W\quad\forall x\in V\setminus\{x_0\}\\
		\text{e, al contempo:}\\
		\forall U\in I(z_0) ~\exists W\in I(y_0)\mid f(y)\in U\quad\forall y\in W\setminus\{y_0\}
	\end{gather*}
	Secondo la prima proposizione, ogni elemento di $V$ viene proiettato tramite $g(x)$ in $W$. Secondo la seconda proposizione, ogni elemento di $W$ viene proiettato tramite $f(x)$ in $U$. Poiché ogni elemento dell'intorno $V$ ha immagine tramite $f(g(x))$ in $U$, il limite dell'intera funzione è $z_0$, ovvero il centro dell'intorno $U$, per definizione. La terza condizione del teorema, inoltre, evita il possibile conflitto dovuto al fatto che il secondo limite non garantisce che $f(y_0)\in U$, per cui $x\in V$, una volta trasportato in $W$ tramite $f(x)$, dev'essere diverso da $y_0$. Questa condizione è sufficiente, ma non necessaria, infatti per le funzioni continue si può elidere.
	% TODO: disegno con gli insiemi
\end{proof}
\begin{examp}
	Un tipico utilizzo di questo teorema è quello di convertire i limiti per $x$ che tende a $0$ in limiti per $x$ che tende a $+\infty$ e viceversa:
	\begin{gather*}
		\lim_{x\to 0^-} e^{\frac{1}{x}}\\
		\text{pongo }y(x)=\frac{1}{x}\\
		\lim_{x\to 0^-} y(x)=\lim_{x\to 0^-} \frac{1}{x}=-\infty\\
		\lim_{y\to -\infty} e^y=0^+
	\end{gather*}
\end{examp}


\subsection{Continuità di funzioni notevoli}
Tutte le funzioni elementari sono continue: trattiamone alcune.
\begin{examp}
	Verifichiamo la continuità nel punto $x=0$ di
	\[
		f(x)=\sqrt{x}
	\]
\end{examp}
\begin{proof}
	$f(0)=0$. Scegliamo ora un'intorno destro di $0$ (poiché la radice non è mai negativa), diciamo $[0,\varepsilon)$. La funzione deve ricadere in questo intorno per $x$ che appartengono a un intorno di $0$ (in questo caso si potrebbe considerare solo un intorno destro, in quanto la funzione radice non è definita per radicandi minori di $0$). Utilizzando il metodo descritto al paragrafo \ref{lim:metodo}:
	\begin{gather*}
		\abs{\sqrt{x}-\sqrt{0}}<\varepsilon\\
		\sqrt{x}<\varepsilon\\
		0\leq x<\varepsilon^2
	\end{gather*}
	L'intervallo ottenuto non è altro che l'intorno che si cercava, per cui la continuità è verificata.
\end{proof}

Dimostriamo ora un teorema utile a verificare la continuità della funzione $y=\abs{x}$:
\begin{teor}
	\label{disug:contrario}
	\[
		\abs{\abs{x}-\abs{x_0}}\leq\abs{x-x_0}\qquad\forall x\in\R
	\]
\end{teor}
\begin{proof}
	Per disuguaglianza triangolare:
	\[
		\abs{a+b}\leq\abs{a}+\abs{b}
	\]
	quindi, se rispettivamente $a=x_0$ o $a=x$ e $b=\pm(x-x_0)$:
	\begin{gather*}
		\begin{cases}
			\abs{x}=\abs{x_0+(x-x_0)}\leq\abs{x_0}+\abs{x-x_0} \\
			\abs{x_0}=\abs{x+(x_0-x)}\leq\abs{x}+\abs{x_0-x}
		\end{cases}\\
		\text{Ovvero (elidendo il termine centrale):}\\
		\begin{cases}
			\abs{x}-\abs{x_0}\leq\abs{x-x_0} \\
			\abs{x_0}-\abs{x}\leq\abs{x_0-x} \\
		\end{cases}
	\end{gather*}
	Notando che i due secondi membri sono uguali e i due primi membri sono opposti, poiché $a<\varepsilon\land a>-\varepsilon\Rightarrow\abs{a}<\varepsilon$:
	\[
		\abs{\abs{x}-\abs{x_0}}\leq\abs{x-x_0}
	\]
\end{proof}

Come preannunciato, questo teorema si utilizza per dimostrare la continuità nel generico $x_0$ (cioè nell'intero dominio) della funzione $y=\abs{x}$:
\begin{examp}
	\[
		f(x)=\abs{x}
	\]
	Dimostrare la continuità nel generico $x_0$ significa dimostrare che
	\[
		\lim_{x\to x_0}\abs{x}=\abs{x_0}
	\]
	Utilizzando il metodo di rappresentazione della definizione di limite tramite valore assoluto:
	\[
		\abs{\abs{x}-\abs{x_0}}\leq\varepsilon\qquad\forall\varepsilon>0
	\]
	Per il teorema \ref{disug:contrario} appena dimostrato:
	\[
		\abs{\abs{x}-\abs{x_0}}\leq\abs{x-x_0}
	\]
	% TODO: spiegare meglio
	Ovvero $x_0-\varepsilon<x<x_0+\varepsilon$, l'intorno di $x$ che si cercava.
\end{examp}
