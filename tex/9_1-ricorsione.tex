%% Copyright (C) 2019-2021 Alessandro Clerici Lorenzini
%
% This work may be distributed and/or modified under the
% conditions of the LaTeX Project Public License, either version 1.3
% of this license or (at your option) any later version.
% The latest version of this license is in
%   http://www.latex-project.org/lppl.txt
% and version 1.3 or later is part of all distributions of LaTeX
% version 2005/12/01 or later.
%
% This work has the LPPL maintenance status `maintained'.
%
% The Current Maintainer of this work is Alessandro Clerici Lorenzini
%
% This work consists of the files listed in work.txt


\section{Definizioni}
Una successione numerica definita ricorsivamente è una successione definita tramite una regola induttiva, cioè un'espressione che esprime un termine in funzione di quello precedente, e un dato iniziale, cioè un termine espresso in maniera assoluta e da cui partono tutte le induzioni.
\begin{examp}
	\[
		a=
		\begin{cases}
			a_{n+1}=2a_n+3 \quad \forall n \geq 0 \\
			a_0=3
		\end{cases}
	\]
	Applicando la regola induttiva al dato iniziale si ottiene il termine in 1, applicandola al termine in 1 si ottiene il termine in 2, et cetera:
	\begin{gather*}
		a_1=2a_0+3=2*3+3=9\\
		a_2=2a_1+3=2*9+3=21\\
		\dots
	\end{gather*}
\end{examp}
Analizzare una successione ricorsiva significa chiedersi se essa si può esprimere in forma esplicita, cioè in una forma che dipenda solo da $n$ e non da un termine precedente, o quantomeno studiarne il comportamento asintotico. Nel caso dell'esempio è sufficiente esprimere in funzione del termine in $0$ i termini successivi a quello in $1$, per ricavarne il pattern:
\begin{gather*}
	a_2=2(2a_0+3)+3=2^2a_0+2*3+3\\
	a_3=2^3a_0+2^2*3+2*3+3\\
\end{gather*}
Ne si ricava che la forma esplicita è
\begin{gather*}
	a_n=2^na_0+(2^{n-1}+2^{n-2}+\dots+2^0)3\\
	\text{ovvero}\\
	2^na_0+3\sum_{k=0}^{n-1} 2^k=2^na_0+3*\frac{2^n-1}{2-1}=2^na_0+3(2^n-1)=\\
	=(2^{n+1}-1)3 \qquad \forall n \geq 0
\end{gather*}
La stessa ricorsività può avere profondità maggiori di $1$, se la regola induttiva contiene più di un termine precedente:
\begin{examp}[successione di Fibonacci]
	\[
		f=
		\begin{cases}
			f_{n+2}=f_{n+1}+f_n \\
			f_0=0               \\
			f_1=1
		\end{cases}
	\]
\end{examp}
In questo caso non esiste una forma che esprima esplicitamente la successione ricorsiva.

% TODO da rivedere
\subsubsection{Interesse composto}
I concetti della ricorsione vedono applicazioni non solo nell'informatica, ma ad esempio anche nell'economia:
Il capitale di un conto a interesse composto all'anno $n$ è dato dalla formula:
\[
	C=
	\begin{cases}
		C_n=C_{n-1}+\alpha*C_{n-1} \\
		C_0=c
	\end{cases}
\]
dove $c$ è il capitale iniziale e $\alpha$ è il tasso di interesse. Si verifica che:
\[
	C_n=C(1+\alpha)^n \qquad \forall n\geq0
\]
\begin{proof}
	Per induzione:
	\begin{itemize}
		\item $n=0$:
		      \[ C_0=C(1+\alpha)^0=C \]
		\item $n \Rightarrow n+1$:
		      \[ C_{n+1}=C(1+\alpha)^{n+1}=C(1+\alpha)^n(1+\alpha)=C_n(1+\alpha) \]
	\end{itemize}
\end{proof}
