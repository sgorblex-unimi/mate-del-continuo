%% Copyright (C) 2019-2021 Alessandro Clerici Lorenzini
%
% This work may be distributed and/or modified under the
% conditions of the LaTeX Project Public License, either version 1.3
% of this license or (at your option) any later version.
% The latest version of this license is in
%   http://www.latex-project.org/lppl.txt
% and version 1.3 or later is part of all distributions of LaTeX
% version 2005/12/01 or later.
%
% This work has the LPPL maintenance status `maintained'.
%
% The Current Maintainer of this work is Alessandro Clerici Lorenzini
%
% This work consists of the files listed in work.txt


\section{Operazioni tra funzioni continue}
Operazioni tra funzioni continue sono continue, in quanto per $x_0$ finiti le funzioni assumono valori finiti, che non si possono quindi tradurre in casi di indecisione. Ad esempio:
\[
	\lim_{x\to x_0} (f(x)+g(x))=\lim_{x\to x_0} f(x)+\lim_{x\to x_0} g(x)=f(x_0)+g(x_0)
\]
che è un numero finito. Questo vale solo parzialmente per i rapporti, in quanto non si può dividere per una funzione che assuma il valore $0$ in $x_0$:
\[
	f(x_0)=\lim_{x \to x_0} f(x_0) \land x_0 \neq 0\Rightarrow \frac{1}{f(x)}=\lim_{x\to x_0} \frac{1}{f(x)}
\]
Per quanto riguarda la composizione:
\[
	\begin{cases}
		\lim_{x\to x_0} g(x)=g(x_0) \\
		\lim_{x\to g(x_0)} f(x)=f(g(x_0))
	\end{cases}\Rightarrow
	\lim_{x\to x_0} (f\circ g)(x)=(f\circ g)(x_0)
\]
che si dimostra utilizzando il teorema \ref{lim:var} per cambiare la variabile del limite della funzione composta:
\[
	\lim_{x\to x_0} (f\circ g)(x)=\lim_{x\to x_0} f(g(x))=\lim_{y\to g(x_0)} f(y)
\]
che per la seconda ipotesi è uguale a $f(g(x_0))$.
