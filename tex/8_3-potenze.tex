%% Copyright (C) 2019-2021 Alessandro Clerici Lorenzini
%
% This work may be distributed and/or modified under the
% conditions of the LaTeX Project Public License, either version 1.3
% of this license or (at your option) any later version.
% The latest version of this license is in
%   http://www.latex-project.org/lppl.txt
% and version 1.3 or later is part of all distributions of LaTeX
% version 2005/12/01 or later.
%
% This work has the LPPL maintenance status `maintained'.
%
% The Current Maintainer of this work is Alessandro Clerici Lorenzini
%
% This work consists of the files listed in work.txt


\section{Serie di potenze}
\begin{defin}
	Si definisce serie di potenze di centro zero\footnote{Una serie di potenze può avere centro diverso da zero. Una serie di potenze di centro $x_0$ si presenta nella forma:
		\[
			\sum_{k=0}^{+\infty} a_n (x-x_0)^n
		\]} una serie nella forma:
	\[
		A(x)=\sum_{n=0}^{+\infty} a_n x^n
	\]
	Con $a_n\in\R\forall n\land x\in\R$
\end{defin}
Per studiare il comportamento di una serie di potenze è innanzitutto utile fare un'osservazione:
\[
	A(0)=\sum_{n=0}^{+\infty} a_n*0^n=
	\begin{cases}
		a_0\quad & n=0    \\
		0 \quad  & n\neq0
	\end{cases}\Rightarrow A(0)\to a_0
\]
In questo caso si sceglie $0^0=1$ per convenzione.

È notevole altresì sottolineare che la serie geometrica è un caso particolare di serie di potenze, con $a_n=1$ costante.

\subsubsection{Raggio di convergenza}
Per studiare la convergenza di una serie di potenze, si studia la convergenza assoluta, dal momento che per il teorema \ref{teor:convass} essa implica la convergenza.
\[
	\sum_{n=0}^{+\infty}\abs{a_n*x^n}=\sum_{n=0}^{+\infty}\abs{a_n}\abs{x}^n
\]
Si applica poi un criterio come quelli del rapporto e della radice:
\[
	\begin{array}{cc}
		\text{Rapporto}                                                                                                 & \text{Radice}                                                            \\[1ex]
		\dfrac{\abs{a_{n+1}}\abs{x}^{n+1}}{\abs{a_n}\abs{x}^n}=\dfrac{\abs{a_{n+1}}}{\abs{a_n}}\abs{x}\to L\abs{x}\quad & \quad\sqrt[n]{\abs{a_n}\abs{x}^n}=\sqrt[n]{\abs{a_n}}\abs{x}\to L\abs{x}
	\end{array}
\]
Quindi, posto che $L$ esista:
\begin{itemize}
	\item Per $L\abs{x}<1\Rightarrow \abs{x}<\frac{1}{L}$ la serie converge assolutamente e quindi converge;
	\item Per $L\abs{x}>1\Rightarrow \abs{x}>\frac{1}{L}$ la serie diverge assolutamente e quindi quantomeno non converge
\end{itemize}
Il numero $\frac{1}{L}=R$ si chiama raggio di convergenza. Si può dimostrare che $A(R)$ non è definita e che, in particolare, $R$ è il minore, in modulo\footnote{Per la precisione, in generale, è il più vicino al centro.}, dei punti in cui $A(x)$ non è definita. Nelle serie geometriche, come è facile verificare, il raggio è $1$.

Quando $L=0$, $R$ non è definito, tuttavia si verifica la prima delle due situazioni sopracitate, quindi la serie converge per ogni $x$. Si dice che $R=+\infty$. Quando $L=+\infty$, $R$ non è definito, tuttavia si verifica la seconda situazione, ad eccezione di $x=0$, nel qual caso la serie converge. Si dice che $R=0$, o che la serie è formale.

\subsubsection{Polinomi tramite serie di potenze}
Un polinomio può essere scritto tramite una serie di potenze, per esempio:
\begin{examp}
	\begin{gather*}
		\sum_{n=0}^{+\infty} a_n x^n= x-\sqrt2 x^3 + 5x^4\\
		\text{con }a_n=
		\begin{cases}
			1\qquad       & n=1           \\
			-\sqrt2\qquad & n=3           \\
			5\qquad       & n=4           \\
			0\qquad       & 1<n<3\lor n>4
		\end{cases}
	\end{gather*}
	In questo caso si ha $R=+\infty$.
\end{examp}


\subsection{Combinazione lineare}
\begin{defin}[combinazione lineare di serie di potenze]
	\label{ser:comblin}
	\begin{equation}
		A\sum_{n=0}^{+\infty}a_n x^n+B\sum_{n=0}^{+\infty} b_nx^n=\sum_{n=0}^{+\infty} (Aa_k+Bb_k)x^n
	\end{equation}
	Da cui deriva
	\begin{equation}
		\label{eq:cperserie}
		c*\sum_{n=0}^{+\infty} a_nx^n=\sum_{n=0}^{+\infty}ca_nx^n
	\end{equation}
\end{defin}
L'operazione ha come elemento neutro la serie nulla e come inverso la serie opposta. Inoltre, per $A=B=1$, se la prima serie ha raggio di convergenza $R_A$ e la seconda $R_B$, ci si aspetta che il raggio di convergenza della loro combinazione lineare sia non inferiore a $\text{min}\{R_A,R_B\}$, e che in caso di convergenza la somma della combinazione sia uguale alla somma delle somme delle due serie. Entrambi questi risultati vengono chiaramente influenzati (ma in modo coerente) se i coefficienti sono diversi da $1$.

\begin{examp}
	Calcolare il raggio di convergenza di
	\[
		\sum_{n=0}^{+\infty}(2^{n+1}+1)x^n=
	\]
	La serie è una combinazione lineare, quindi:
	\[
		=2\sum_{n=0}^{+\infty}2^nx^n+\sum_{n=0}^{+\infty}x^n
	\]
	È possibile applicare una sostituzione, dal momento che $2^nx^n=(2x)^n$ e $t=2x$ non dipende da $n$:
	\[
		=2\sum_{n=0}^{+\infty}t^n+\sum_{n=0}^{+\infty}x^n
	\]
	La prima delle due serie converge a $\frac{1}{1-t}=\frac{1}{1-2x}$ con raggio $\frac{1}{2}$, mentre la seconda converge a $\frac{1}{1-x}$ con raggio $1$. La combinazione lineare, nonché serie iniziale, converge quindi a
	\[
		\frac{1}{1-2x}+\frac{1}{1-x}
	\]
	con raggio $\abs{x}<\frac{1}{2}$.
\end{examp}


\subsection{Derivazione}
La derivata di una serie di potenze viene così definita:
\begin{defin}
	\[
		D\left(\sum_{n=0}^{+\infty} a_nx^n\right):=\sum_{n=0}^{+\infty}(n+1)a_{n+1}x^n
	\]
\end{defin}
La definizione è coerente con il concetto di derivata, infatti:
\[
	D(a_0+a_1x+a_2x^2+a_3x^3+\dots)=0+a_1+a_2x+a_3x^2+\dots
\]
Vale inoltre la seguente relazione, che non dimostriamo:
\[
	D\left(\sum_{n=0}^{+\infty} a_nx^n\right)=\sum_{n=0}^{+\infty}D(a_nx^n)=\sum_{n=0}^{+\infty}a_nnx^{n-1}
\]
di cui l'ultima espressione è equivalente alla definizione, in quanto per $n=0$ il coefficiente di $x^{-1}$ è $0$.

\subsubsection{Somma e raggio della derivata}
Per quanto riguarda la somma della derivata, si dimostra che:
\begin{teor}
	Se una serie di potenze di raggio $R$ converge ad $A(x)$, allora la sua derivata converge ad $A'(x)$:
	\[
		\sum_{n=0}^{+\infty} a_n x^3=A(x)\Rightarrow\sum_{n=0}^{+\infty} a_n 2x^2=A'(x)\qquad\forall x\in(-R,R)
	\]
	Con $R$ comune.
\end{teor}
Benché la dimostrazione vada oltre gli strumenti precedentemente introdotti, si dimostra in seguito l'uguaglianza dei raggi:
\begin{proof}
	Applicando il criterio del rapporto all'argomento della derivata:
	\[
		\frac{1}{R'}=\lim_{n\to+\infty}\frac{b_{n+1}}{b_n}=\lim_{n\to+\infty}\frac{n+2}{n+1}*\frac{a_{n+2}}{a_{n+1}}=\frac{1}{R}
	\]
	Nell'ultimo prodotto, il primo fattore tende a $1$, mentre il secondo è il criterio del rapporto applicato alla serie non derivata e traslato di $+1$: tende pertanto allo stesso limite, cioè $\frac{1}{R}$.
\end{proof}

\begin{examp}
	\label{ex:dergeom}
	Calcolare la somma della seguente serie di potenze:
	\[
		\sum_{n=0}^{+\infty}(n+1)x^n=D\left(\sum_{n=0}^{+\infty}x^n\right)=D\left(\frac{1}{1-x}\right)=\frac{1}{(1-x)^2}
	\]
\end{examp}

\subsubsection{Derivata $m$-esima}
Calcolare le derivate successive alla prima significa calcolare la derivata di una derivata. Ad esempio:
\begin{examp}
	\[
		D^2\left(\sum_{n=0}^{+\infty}a_nx^n\right)=D\left(D\left(\sum_{n=0}^{+\infty}a_nx^n\right)\right)=D\left(\sum_{n=0}^{+\infty}(n+1)a_{n+1}x^n\right)
	\]
	Chiamando il coefficiente di $x^n$ $b_n$:
	\[
		=\sum_{n=0}^{+\infty}(n+1)b_{n+1}x^n=\sum_{n=0}^{+\infty}(n+1)(n+2)a_{n+2}x^n
	\]
\end{examp}

La derivata $m$-esima di una serie di potenze è espressa dalla seguente formula generalizzata:
\begin{equation}
	D^m\left(\sum_{n=0}^{+\infty}a_nx^n\right)=\sum_{n=0}^{+\infty}(n+1)(n+2)\dots(n+m)a_{n+m}x^n
\end{equation}
Questa scrittura viene compattata con il coefficiente binomiale, dividendo entrambi i membri per $m!$:
\[
	\frac{1}{m!}D^m\left(\sum_{n=0}^{+\infty}a_nx^n\right)=\sum_{n=0}^{+\infty}\binom{n+m}{m}a_{n+m}x^n
\]

Applicando la formula alla serie geometrica di potenze vale:
\[
	\frac{1}{(1-x)^{m+1}}=\frac{1}{m!}D^m\left(\sum_{n=0}^{+\infty}x^n\right)=\sum_{n=0}^{+\infty}\binom{n+m}{m}x^n
\]
Dove il primo membro è la derivata $m$-esima della somma divisa per $m!$.


\subsection{Prodotto di Cauchy}
Il prodotto di Cauchy (o di convoluzione) si pone lo scopo di definire il prodotto di due serie.
\[
	\left(\sum_{n=0}^{+\infty} a_nx^n\right)\left(\sum_{n=0}^{+\infty}b_nx^n\right)=\sum_{n=0}^{+\infty}\left(\sum_{k=0}^na_kb_{n-k}\right)x^n
\]
Infatti, in modo esplicito:
\[
	(a_0+a_1x+a_2x^2+a_3x^3+\dots)(b_0+b_1x+b_2x^2+b_3x^3+\dots)
\]
Ai termini di $n$-esimo grado contribuiscono solo i coefficienti la cui somma è $n$:
\[
	a_0b_0+(a_0b_1+a_1b_0)x+(a_0b_2+a_1b_1+a_2b_0)x^2+\dots
\]
da cui l'espressione di cui sopra.

Vale inoltre il seguente teorema
\begin{teor}
	Il prodotto di Cauchy di due serie convergenti converge al prodotto delle somme:
	\[
		\begin{cases}
			\displaystyle\sum_{n=0}^{+\infty}a_nx^n=A(x)\quad\text{per }\abs{x}<R_A \\[2ex]
			\displaystyle\sum_{n=0}^{+\infty}b_nx^n=B(x)\quad\text{per }\abs{x}<R_B
		\end{cases}\Rightarrow
		\sum_{n=0}^{+\infty}\left(\sum_{k=0}^na_kb_{n-k}\right)x^n=A(x)B(x)
	\]
	con raggio $\abs{x}<\text{min}\{R_A,R_B\}$
\end{teor}
\begin{examp}
	\[
		\left(\sum_{n=0}^{+\infty}x^n\right)^2=\sum_{n=0}^{+\infty}\left(\sum_{k=0}^n 1*1\right)x^n=\sum_{n=0}^{+\infty}(n+1)x^n
	\]
	Che come già visto all'esempio \ref{ex:dergeom} non è altro che la derivata della serie geometrica ed è quindi uguale a $\frac{1}{(1-x)^2}$.
\end{examp}

\begin{examp}
	\begin{equation}
		\label{ex:prodcauchy}
		x^p\sum_{n=0}^{+\infty}a_nx^n=\sum_{n=0}^{+\infty}a_nx^{n+p}=\sum_{n=p}^{+\infty}a_{n-p}x^n
	\end{equation}
	In questa soluzione il prodotto è stato interpretato come una combinazione lineare di una costante con una successione (\ref{eq:cperserie} alla definizione \ref{ser:comblin}). Il prodotto può essere interpretato anche come il prodotto di Cauchy della serie di potenze di $a_n$ con la serie:
	\[
		\sum_{n=0}^{+\infty}\delta_{np}x^n\qquad\text{con }\delta_{np}=
		\begin{cases}
			1\quad & n=p     \\
			0\quad & n\neq p
		\end{cases}
	\]
	In tal caso il prodotto equivale a
	\[
		\left(\sum_{n=0}^{+\infty}\delta_{np}x^n\right)\left(\sum_{n=0}^{+\infty}a_nx^n\right)=\sum_{n=0}^{+\infty}\left(\sum_{k=0}^n \delta_{kp}a_{n-k}\right)
	\]
	E dal momento che
	\[
		\sum_{k=0}^n \delta_{kp}a_{n-k}=
		\begin{cases}
			0\quad       & n<p     \\
			a_{n-p}\quad & n\geq p
		\end{cases}
	\]
	Il risultato equivale a quello ottenuto nella \ref{ex:prodcauchy}.
\end{examp}


\subsection{Serie di Taylor}
Gli sviluppi di Taylor permettono di approssimare una funzione attorno a un centro, a meno di un errore che dipende dall'ordine $N$:
\[
	f(x)=\sum_{n=0}^N\left[\frac{1}{n!}D^nf(0)x^n\right]+o(x^N)\qquad x\to0
\]
Dove
\[
	P_n=\sum_{n=0}^N\left[\frac{1}{n!}D^nf(0)x^n\right]
\]
è detto polinomio di Taylor. Una particolare classe di funzioni, dette analitiche, può essere espressa tramite un polinomio di Taylor di ordine infinito, chiamato serie di Taylor:
\begin{defin}
	Si dice serie di Taylor di una funzione $f$ liscia con centro $x_0$ la serie
	\begin{equation}
		\sum_{n=0}^{+\infty}\frac{1}{n!}D^nf(x_0)(x-x_0)^n
	\end{equation}
\end{defin}
Una serie di Taylor di centro $0$ si dice anche serie di McLaurin. Le serie di Taylor sono serie di potenze il cui raggio è l'intervallo in cui la funzione è liscia (cioè derivabile infinite volte). Per $x$ interno al raggio di convergenza, la serie è uguale alla funzione, senza bisogno di specificare alcun errore.

\subsubsection{Serie di Taylor notevoli}
\begin{itemize}
	\item $f(x)=e^x$. Il limite dello sviluppo di McLaurin per $N\to+\infty$ è per definizione la ridotta della serie di Taylor:
	      \[
		      \lim_{N\to+\infty}\sum_{n=0}^N\frac{x^n}{n!}=\sum_{n=0}^{+\infty}\frac{x^n}{n!}
	      \]
	      Come è facile dimostrare (criterio del rapporto) il raggio di convergenza della serie è $+\infty$. Il teorema \vref{tay:restolagrange} (resto secondo Lagrange dello sviluppo di Taylor) dimostra che
	      \[
		      \exists c_{x,N}\in(-x,x)\setminus\{0\}\mid R_n=\frac{e^{c_{x,N}}}{(N+1)!}x^{N+1}
	      \]
	      Essendo $R_n=f(x)-P_n$:
	      \[
		      e^x-\sum_{n=0}^N\frac{x^n}{n!}=e^{c_{x,N}}\frac{x^{N+1}}{(N+1)!}
	      \]
	      Poiché $c_{x,N}\leq \abs{x}$:
	      \[
		      e^x-\sum_{n=0}^N\frac{x^n}{n!}\leq e^{\abs{x}}\frac{x^{N+1}}{(N+1)!}
	      \]
	      Per $N\to+\infty$ il secondo membro tende a $0$, quindi:
	      \[
		      \sum_{n=0}^N\frac{x^n}{n!}\to e^x
	      \]
	      ovvero
	      \begin{equation}
		      e^x=\sum_{n=0}^{+\infty}\frac{x^n}{n!}\qquad\forall x\in\R
	      \end{equation}
	\item $f(x)=\ln(1+x)$ Con una dimostrazione simile alla precedente, si dimostra che
	      \begin{equation}
		      \ln(1+x)=\sum_{n=1}^{+\infty}\frac{(-1)^{n-1}}{n}x^n\qquad\forall \abs{x}<1
	      \end{equation}
	\item $f(x)=(1+x)^\alpha$. La serie di Taylor ha coefficiente $\binom{\alpha}{n}$. Applicando il criterio del rapporto si ricava il raggio:
	      \begin{gather*}
		      \abs{\frac{a_{n+1}}{a_n}}=\abs{\frac{\binom{\alpha}{n+1}}{\binom{\alpha}{n}}}=\\
		      =\abs{\frac{\alpha(\alpha-1)\dots(\alpha-(n+1)+1)}{(n+1)!}\cdot\frac{n!}{\alpha(a-1)\dots(\alpha-n+1)}}=\\
		      =\abs{\frac{\alpha-n}{n+1}}\to1
	      \end{gather*}
	      Per quanto riguarda la somma, si dimostra il caso della serie geometrica, cioè di $\alpha=-1$ e $x=-x$:
	      % TODO: spiegare meglio
	      \begin{gather*}
		      \sum_{n=0}^{+\infty}\binom{-1}{n}(-x)^n=\sum_{n=0}^{+\infty}\binom{-1}{n}(-1)^nx^n\\
		      \text{studiando il coefficiente:}\\
		      \binom{-1}{n}(-1)^n=\frac{(-1)(-2)\dots(-1-n+1)}{n!}*(-1)^n=\\=\frac{(-1)^n n!}{n!}*(-1)^n=1\quad\forall n
	      \end{gather*}
	      Quindi la serie è uguale alla serie geometrica di ragione $x$, quindi, come già visto, vale:
	      \[
		      \frac{1}{1-x}=\sum_{n=0}^{+\infty}x^n
	      \]
	      e quindi
	      \[
		      (1+(-x))^{-1}=\sum_{n=0}^{+\infty}\binom{-1}{n}(-x)^n
	      \]
	      Più in generale, vale
	      \begin{equation}
		      (1+x)^\alpha=\sum_{n=0}^{+\infty}\binom{\alpha}{n} x^n\qquad\forall \abs{x}<1
	      \end{equation}
	% TODO: aggiungere seno e coseno
\end{itemize}
