%% Copyright (C) 2019-2021 Alessandro Clerici Lorenzini
%
% This work may be distributed and/or modified under the
% conditions of the LaTeX Project Public License, either version 1.3
% of this license or (at your option) any later version.
% The latest version of this license is in
%   http://www.latex-project.org/lppl.txt
% and version 1.3 or later is part of all distributions of LaTeX
% version 2005/12/01 or later.
%
% This work has the LPPL maintenance status `maintained'.
%
% The Current Maintainer of this work is Alessandro Clerici Lorenzini
%
% This work consists of the files listed in work.txt


\section{La derivata}
Il calcolo differenziale è lo strumento fondamentale alla base dello studio di funzione, il procedimento che permette di ricavare il grafico probabile, cioè approssimativo, di una funzione. Il problema che si pone per introdurre il concetto di derivata è quello di determinare la tangente a una funzione in un punto $x_0$. La tangente viene ricavata mettendo insieme un elemento geometrico, quello della secante per due punti, a uno algebrico, il limite che porta i due punti a coincidere.

Posto che i due punti in cui la retta seca la funzione siano quelli di coordinate $(x_0, f(x_0))$ e $(x+h, f(x+h))$ (dove $h$ è una distanza in ascissa), la retta che passa per essi è data dal fascio di rette passanti per il punto di ascissa $x_0$:
\begin{equation}
	\label{fasciof}
	y-f(x_0)=m(x-x_0)
\end{equation}
in cui $m$ assume il valore $\frac{\Delta y}{\Delta x}$, detto anche rapporto incrementale:
\[
	\frac{f(x_0+h)-f(x_0)}{h}
\]

La tangente in questa forma esiste se e solo se esiste ed è finito il seguente limite, che prende il nome di derivata (o derivata prima):
\begin{defin}[Derivata]
	Si dice derivata di $f(x)$ nel punto $x_0$ il limite, se esiste:
	\label{defin:deriv}
	\[
		f'(x_0)=\lim_{h\to0} \frac{f(x_0+h)-f(x_0)}{h}
	\]
	La funzione $f$ si dice derivabile se tale limite esiste e non derivabile se esso non esiste.
\end{defin}
La tangente in $x_0$ assumerà quindi l'espressione
\[
	y=f(x_0)+f'(x_0)(x-x_0)
\]

La tangente in $x_0$ di una funzione viene anche definita come la retta che meglio approssima la funzione in $x_0$. In questa prospettiva studiamo la distanza per $x$ che tende a $x_0$ tra i valori che assume la funzione e i valori che assume il fascio di rette dell'espressione \ref{fasciof}:
\[
	\lim_{x\to x_0} f(x)-[f(x_0)+m(x-x_0)]
\]
Notiamo che l'ultimo addendo tende a $0$, mentre il rimanente $f(x)-f(x_0)$ tende a $0$ se e solo se la funzione è continua in $x_0$. Questo significa che qualunque secante del fascio approssima la funzione, se essa è continua, per $x$ che tende a $x_0$. Per trovare quale delle infinite rette del fascio è quella che meglio approssima la funzione, si cerca un valore per $m$ tale per cui la distanza studiata sia $o(x-x_0)$:
\[
	f(x)-[f(x_0)+m(x-x_0)]=o(x-x_0)\qquad x\to x_0
\]
Risolvendo secondo la definizione di $o$:
\begin{gather*}
	\lim_{x\to x_0}\frac{f(x)-f(x_0)-m(x-x_0)}{x-x_0}=0\\
	\lim_{x\to x_0}\frac{f(x)-f(x_0)}{x-x_0}-m=0\\
	\text{Ovvero, poiché m non dipende da $x$:}\\
	m=\lim_{x\to x_0} \frac{f(x)-f(x_0)}{x-x_0}
\end{gather*}
che non è altro che la definizione di derivata, in quanto nella definizione \ref{defin:deriv} $h=x-x_0$.


\subsection{Sviluppo della funzione incrementata}
\begin{teor}
	\label{derivo}
	\[
		f'(x_0)=m \iff f(x_0+h)=f(x_0)+mh+o(h)\qquad h\to0
	\]
	dove la prima proposizione implica anche la derivabilità.
\end{teor}
\begin{proof}
	\begin{gather*}
		\lim_{h\to0} \frac{f(x_0+h)-f(x_0)}{h}=m\\
		f(x_0+h)-f(x_0)-mh=o(h)\qquad h\to0\\
		f(x_0+h)=f(x_0)+mh+o(h)
	\end{gather*}
	Poiché i passaggi sono tutti invertibili (e inoltre la derivata, se esiste, è unica), vale anche l'implicazione inversa.
\end{proof}

\begin{corol}
	Se una funzione è derivabile in un punto, allora essa è continua in quel punto.
\end{corol}
\begin{proof}
	Utilizzando la derivata così come espressa nel teorema \ref{derivo}:
	\[
		f(x_0+h)=f(x_0)+mh+o(h)\qquad h\to0
	\]
	Poiché gli ultimi due addendi tendono a $0$, l'espressione equivale a
	\begin{align*}
		f(x_0+h)          & = f(x_0)+o(1)\qquad h\to0 \\
		f(x_0+h) - f(x_0) & = o(1)\qquad h\to0
	\end{align*}
	che è equivalente alla definizione \vref{defin:continua} di funzione continua nel punto $x_0$.
\end{proof}


\subsection{Tipi di non derivabilità}
Esistono casi in cui, nonostante la funzione sia continua in un punto, la sua derivata non esiste:
\begin{itemize}
	\item se i limiti destro e sinistro del rapporto incrementale esistono e valgono infiniti discordi, il punto di non derivabilità si dice cuspide;
	\item se i limiti destro e sinistro del rapporto incrementale esistono e valgono infiniti concordi, il punto di non derivabilità è un flesso a tangente verticale;
	\item se i limiti destro e sinitro esistono, sono diversi e almeno uno dei due è finito, il punto di non derivabilità si dice punto angoloso.
\end{itemize}
\begin{examp}
	La funzione $f(x)=\abs{x}$ ha in $x=0$ un punto angoloso:
	\[
		\lim_{h\to 0} \frac{\abs{0+h}-\abs{0}}{h} = \lim_{h\to 0} \frac{\abs{h}}{h}
	\]
	Questo limite non esiste, ma i limiti destro e sinistro valgono rispettivamente $+1$ e $-1$:
	\[
		\lim_{h\to 0^+} \frac{\abs{h}}{h}=1\qquad\lim_{h\to 0^-} \frac{\abs{h}}{h}=-1
	\]
\end{examp}

\begin{examp}
	Per analizzare un caso un po' più complicato, si prenda come esempio la funzione $f(x)=x^\alpha$, con $\alpha>0$, anche questa volta nel punto di ascissa $0$:
	\[
		\lim_{h\to0} \frac{f(0+h)-f(0)}{h} = \lim_{h\to0} \frac{h^\alpha}{h}=\lim_{h\to0} h^{\alpha-1}
	\]
	Questo limite assume diversi valori a seconda del valore di $\alpha$, in particolare:
	\[
		\lim_{h\to0} h^{\alpha-1}=
		\begin{cases}
			0^+ & \alpha>1 \\
			1   & \alpha=1
		\end{cases}
	\]
	Mentre per $0<\alpha<1$ i limiti destro e sinistro non coincidono:
	\[
		\lim_{h\to0^+} h^{\alpha-1}=+\infty\qquad\lim_{h\to0^-} h^{\alpha-1}=-\infty
	\]
	Nel primo e nel secondo caso la derivata esiste, mentre nell'ultimo caso i limiti destro e sinistro del rapporto incrementale assumono valori infiniti e discordi. Si ha quindi un punto di cuspide.
\end{examp}
