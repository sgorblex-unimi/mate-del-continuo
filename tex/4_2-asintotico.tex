%% Copyright (C) 2019-2021 Alessandro Clerici Lorenzini
%
% This work may be distributed and/or modified under the
% conditions of the LaTeX Project Public License, either version 1.3
% of this license or (at your option) any later version.
% The latest version of this license is in
%   http://www.latex-project.org/lppl.txt
% and version 1.3 or later is part of all distributions of LaTeX
% version 2005/12/01 or later.
%
% This work has the LPPL maintenance status `maintained'.
%
% The Current Maintainer of this work is Alessandro Clerici Lorenzini
%
% This work consists of the files listed in work.txt


\section{\texorpdfstring{$\sim$ (asintotico)}{Asintotico}}
Il simbolo di asintotico viene così definito:
\begin{defin}
	\[
		a_n\sim b_n:\qquad\frac{a_n}{b_n}\to1
	\]
\end{defin}


\subsection{Proprietà}
La relazione gode di proprietà riflessiva (poiché $\frac{a_n}{a_n}=1$), simmetrica (poiché il reciproco di una successione che tende a 1 tende a 1) e transitiva:
\begin{proof}
	\begin{gather*}
		a_n\sim b_n\land b_n\sim c_n\Rightarrow a_n\sim c_n\\
		\frac{a_n}{c_n}=\frac{a_n}{b_n}*\frac{b_n}{c_n}\to1
	\end{gather*}
	In quanto i due fattori tendono a 1 per ipotesi.
\end{proof}
L'asintotico è quindi una relazione di equivalenza. Si dimostra che tutti gli elementi di una classe di equivalenza hanno lo stesso limite:
\begin{teor}
	\[
		\begin{cases}
			a_n\sim b_n \\
			b_n\to l
		\end{cases}\Rightarrow a_n\to l
	\]
\end{teor}
\begin{proof}
	\[
		a_n=\frac{a_n}{b_n}
	\]
	Che tende a $l$ in quanto il primo fattore tende a $1$ per definizione di asintotico e il secondo tende a $l$ per ipotesi.
\end{proof}


\subsection{Operazioni}
\begin{prop}
	\[
		\begin{cases}
			a_n\sim b_n \\
			A_n\sim B_n
		\end{cases}
		\begin{aligned}
			 & \Rightarrow\frac{a_n}{A_n}\sim\frac{b_n}{B_n}             \\
			 & \Rightarrow a_n\cdot A_n\sim b_n\cdot B_n                 \\
			 & \Rightarrow a_n^\alpha\sim b_n^\alpha\quad\forall\alpha>0
		\end{aligned}
	\]
\end{prop}
\begin{proof}
	Si dimostra la prima implicazione, in quanto le altre ne seguono direttamente. Traducendo la tesi in simboli:
	\[
		\frac{a_n/A_n}{b_n/B_n}\to1
	\]
	Semplificando l'espressione:
	\[
		\frac{a_n}{b_n}\cdot \frac{B_n}{A_n}
	\]
	Che tende a $1$ in quanto prodotto di due successioni tendenti a $1$ per ipotesi.
\end{proof}
Somme ed esponenziali non godono delle stesse proprietà, mentre per i logaritmi si pone un'ulteriore condizione:
% TODO: aggiungere controesempio per necessaria seconda condizione: n/(n+1) e (n+1)/n
\begin{prop}
	\[
		\begin{cases}
			a_n\sim b_n   \\
			b_n\to+\infty \\
			0<q\neq1
		\end{cases}\Rightarrow\log_qa_n\sim\log_qb_n
	\]
\end{prop}
\begin{proof}
	\[
		\frac{\log_qa_n}{\log_qb_n}=\frac{\log_q\left(\dfrac{a_n}{b_n}\cdot b_n\right)}{\log_qb_n}=\frac{\log_q(a_n/b_n)+\log_qb_n}{\log_qb_n}=\frac{\log_q(a_n/b_n)}{\log_qb_n}+1
	\]
	L'ultima espressione è la somma di un addendo che tende a $0$ e $1$, quindi il limite è $1$.
\end{proof}

La seguente è una proprietà dell'asintotico, utile in particolare in alcune dimostrazioni, che gli permette di essere scambiato con l'$o$ piccolo:
\begin{prop}
	\[
		a_n\sim b_n\iff a_n=b_n+o(b_n)
	\]
\end{prop}
\begin{proof}
	\begin{gather*}
		\frac{a_n}{b_n}\to1\\
		\text{quindi}\\
		\frac{a_n}{b_n}-1\to0\\
		\frac{a_n-b_n}{b_n}\to0\\
		\text{ergo}\\
		a_n-b_n=o(b_n)\\
		a_n=b_n+o(b_n)
	\end{gather*}
\end{proof}
\begin{examp}
	\begin{gather*}
		n^2+n\log n+\frac{n^6}{2^n}=\\
		n^2+o(n^2)+o(1)=n^2+o(n^2)+o(n^2)=n^2+o(n^2)\sim n^2\to+\infty
	\end{gather*}
\end{examp}

\begin{teor}
	\[
		a_n\sim b_n\Rightarrow o(a_n)\equiv o(b_n)
	\]
\end{teor}
\begin{proof}
	Si dimostra solo l'implicazione $o(a_n)=o(b_n)$. Poiché l'ipotesi è simmetrica, anche la tesi lo sarà. Traducendo la tesi secondo la definizione:
	\[
		\frac{o(a_n)}{b_n}=\frac{a_n\cdot o(1)}{b_n}=\frac{a_n}{b_n}\cdot o(1)
	\]
	Che tende a $0$ in quanto prodotto di un fattore che tende a $1$ (per ipotesi) e uno che tende a $0$ (per definizione).
\end{proof}
