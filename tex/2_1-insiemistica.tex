\section{Concetti di insiemistica}
In questa sezione si applicano a $\R$ e ai suoi sottoinsiemi concetti algebrici generalizzabili a un'ampia classe di insiemi.

\subsection{Maggioranti e minoranti}
\begin{defin}
	Posto $\emptyset\neq A\subset\R$ si dice insieme dei maggioranti di $A$ e si indica con $A^*$:
	\[
		A^*:=\{M\in\R\mid\forall a\in A\quad a\leq M\}
	\]
\end{defin}
% TODO: grafico retta reali
E, analogamente:
\begin{defin}
	Posto $\emptyset\neq A\subset\R$ si dice insieme dei minoranti di $A$ e si indica con $A_*$:
	\[
		A_*=:\{m\in\R\mid\forall a\in A\quad a\geq m\}
	\]
\end{defin}
Si osserva che $M\in A^*\land M'\geq M\Rightarrow M'\in A^*$ (se un numero $M'$ è maggiore o uguale a un maggiorante $M$ esso stesso è un maggiorante). Cosa analoga avviene per i minoranti. Ciò significa che i maggioranti e i minoranti sono descritti da intervalli illimitati, i primi del tipo $[M,+\infty)$ e i secondi del tipo $(-\infty,m]$.

Si ricorda che la negazione della proposizione che definisce i maggioranti (e allo stesso modo quella dei minoranti) è:
\begin{propo}
	\[
		M\notin A^*\iff\exists a\in A\mid a>M
	\]
\end{propo}


\subsection{Massimo e minimo}
\begin{defin}
	Si dice massimo di un insieme $\emptyset\neq A\subseteq\R$ e si indica con $\max A$:
	\[
		\max A=M\quad\Leftrightarrow\quad M\in A\land M\in A^*
	\]
\end{defin}
E, analogamente:
\begin{defin}
	Si dice minimo di un insieme $\emptyset\neq A\subseteq\R$ e si indica con $\min A$:
	\[
		\min A=m\quad\Leftrightarrow\quad m\in A\land m\in A_*
	\]
\end{defin}

Quando esiste, il massimo (così come il minimo) è unico:
\begin{proof}
	\begin{gather*}
		M,M'\in A^*\cap A\\
		\begin{cases}
			M\in A \\
			M'\in A^*
		\end{cases}\Rightarrow M\leq M'\\
		\begin{cases}
			M'\in A \\
			M\in A^*
		\end{cases}\Rightarrow M'\leq M
	\end{gather*}
	Ciò è possibile se e solo se $M=M'$
\end{proof}


\subsection{Estremi}
\begin{defin}
	Si dice estremo superiore di un insieme $\emptyset\neq A\subseteq\R$ e si indica con $\sup A$:
	\[
		\sup A:=
		\begin{cases}
			\min A^*\qquad & \text{se }A^*\neq\emptyset \\
			+\infty\qquad  & \text{se }A^*=\emptyset
		\end{cases}
	\]
\end{defin} E,
analogamente:
\begin{defin}
	Si dice estremo inferiore di un insieme $\emptyset\neq A\subseteq\R$ e si indica con $\inf A$:
	\[
		\inf A:=
		\begin{cases}
			\max A_*\qquad & \text{se  }A_*\neq\emptyset \\
			-\infty\qquad  & \text{se }A_*=\emptyset
		\end{cases}
	\]
\end{defin}
Si osservi che per dimostrare che un numero $S'$ non è un maggiorante è sufficiente:
\begin{gather*}
	\sup A=:S\\ S'<S\Rightarrow S'\notin A^*\\
	\text{ovvero, ponendo un $\varepsilon>0$ arbitrario:}\\
	S' = S-\varepsilon\notin A^*
\end{gather*}

Per gli insiemi definiti da un intervallo che non contiene l'estremo superiore, ossia insiemi del tipo $[a,b)$ oppure $(a,b)$, il massimo non esiste (mentre non esiste il minimo di insiemi definiti da intervalli che non contengono l'estremo inferiore):
\begin{proof}
	\begin{gather*}
		A=[a,b)\\
		b\notin A\Rightarrow b\neq M\\
		\text{pongo una variabile arbitraria } \varepsilon>0\\
		b-\varepsilon\notin A^*\Leftarrow b-\frac{\varepsilon}{2}>b-\varepsilon\land b-\frac{\varepsilon}{2}\in A
	\end{gather*}
	Il che vale per ogni valore possibile di $\varepsilon$ e quindi implica l'impossibilità dell'esistenza di un massimo.
\end{proof}
Analoga, ovviamente, è la dimostrazione nel caso del minimo.
